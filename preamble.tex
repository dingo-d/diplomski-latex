%%% Paketi

%% Za pdfTeX
\usepackage{ifpdf}
\usepackage{ifthen}

%% Osnovni paketi
\usepackage{ucs}
\usepackage[utf8x]{inputenc}
\usepackage[T1]{fontenc}
%\usepackage{fourier}% fontovi -Utopia ili neki drugi ja koristim garamond
%\usepackage[utopia]{mathdesign}
%\renewcommand*{\rmdefault}{mdugm}
%\usepackage{pxfonts}
\usepackage{tgtermes}
%\usepackage{emerald}
%\usepackage[urw-garamond]{mathdesign}
%\usepackage{lucidabr}
\usepackage[kerning=true,babel=true]{microtype}
\usepackage{textcomp}
\usepackage{setspace}
\usepackage[headheight=15pt]{geometry}
\usepackage{fancyhdr}
\usepackage[usenames,dvipsnames,svgnames]{xcolor}
\usepackage{natbib}
\setcitestyle{square,numbers,comma}
\usepackage{cancel} % paket za križanje članova

%% Ostali paketi

% Matematika
\usepackage{amsmath}
%\usepackage{amssymb} javlja error kad se koristi amssymb, provjeriti to!
\usepackage{amsfonts}
\usepackage{array}
\usepackage{multirow}
\usepackage{mathtools}
\usepackage{amsthm}
\usepackage{mathrsfs}
\usepackage{wasysym} %astro objekti

% Vizuali
\usepackage{lettrine} % za veliko početno slovo
\renewcommand{\LettrineFontHook}{\color[gray]{0.4}}
%\usepackage{tikz}
%\usetikzlibrary{arrows,patterns,plotmarks,backgrounds,shapes,shadows,,arrows}

%% Boje

%\definecolor{zut}{RGB}{253,200,0}
%\definecolor{nar}{RGB}{240,95,0}
%\definecolor{pla}{RGB}{134,206,255}
\definecolor{DarkGray}{RGB}{92,92,92}

% Izgled dokumenta (layout)
\usepackage[margin=10pt,font=small,labelfont=bf, labelsep=endash]{caption}
%\usepackage[center]{caption} % prije subfig
\usepackage{subfig} % nekoliko figura u jednoj (zamjena za subfigure)
\usepackage{pdfpages} % uključivanje pdf-a
\usepackage{titlesec}
\usepackage{wrapfig}
\usepackage[nottoc]{tocbibind}
\usepackage{colortbl}
\usepackage{booktabs}
\setlength{\heavyrulewidth}{0.2em}

% Korisno
\usepackage{lastpage}
\usepackage[utopia]{quotchap}
\renewcommand{\chapnumfont}{\fontsize{100}{130}\selectfont\color{chaptergrey}}

%% Babel
\usepackage[croatian]{babel}
\usepackage[babel]{csquotes}
%\MakeAutoQuote{«}{»}
\usepackage[hyphens]{url}

%% Za fancyhdr da može koristiti velika slova
\makeatletter
\adddialect\l@CROATIAN\l@croatian
\makeatother


%% hyperref, xcolor mora biti loadan prije toga
\usepackage{graphicx}  % Za korištenje grafike
\usepackage{pstricks}
\usepackage{pdflscape}
\usepackage[hyperindex, colorlinks=true, linkcolor=DarkGray, citecolor=DarkGray, urlcolor=DarkGray, plainpages=false, pdfpagelabels, hypertexnames=false, unicode]{hyperref}
%%Dodaci za pofarbane citate i linkove
% colorlinks=true, urlcolor=DarkOrchid, citecolor=ForestGreen, linkcolor=MidnightBlue, hidelinks %

%\usepackage[all]{hypcap}

\usepackage[toc,page]{appendix}

\usepackage[bottom]{footmisc}

%% tablice - sideways

\usepackage{rotating}



%%% Custom komande

\newcommand{\ornamentChapter}{\begin{center}\begin{Large}\textorn{2}\end{Large}\end{center}}
%% Centriranje kolumne redaka u multiline okruženju
\renewcommand\multirowsetup{\centering}

%% Stvaranje neke vrste centra širine stupca 1.5cm
\newcolumntype{D}{>{\centering}p{1.5cm}}

%% Bilješke u margini sa strane
\newcommand{\note}[1]{\marginpar{%
  \vskip-\baselineskip % malo podizanje marginpar
  \raggedright\footnotesize
  \color{red}{\itshape\hrule\smallskip#1}\par\smallskip\hrule}}
\newcommand{\nonotes}{\renewcommand{\note}[1]{}}

%% Strong %%
\DeclareRobustCommand{\strong}[1]{%
    \textbf{#1}%
}

%% Zahvala/posveta %%
\DeclareRobustCommand{\dedicace}[1]{%
    \clearemptydoublepage
    \thispagestyle{empty}
    \vspace*{\stretch{1}}\par
    {\begin{flushright}#1\end{flushright}\par}
    \vspace*{\stretch{2}}
}

%% Na paragrafu (granici između dvije grupe paragrafa)%%
\DeclareRobustCommand{\surparagraph}{%
    \par\medskip
}

%% Korištenje klasičnih oznaka (ne garamond) za kaligrafski L i O 
\DeclareMathAlphabet{\xcal}{OMS}{cmsy}{m}{n}
\newcommand{\Lie}{\xcal{L}}
\newcommand{\bigo}{\xcal{O}}

%% Redefinicija derivacija radi jednostavnosti
\newcommand{\fd}[2]{\frac{d{#1}}{d{#2}}} %first derivative%
\newcommand{\sd}[2]{\frac{d^2{#1}}{d{#2}^2}} %second derivative%
\newcommand{\fpd}[2]{\frac{\partial{#1}}{\partial{#2}}} %first partial derivative%
\newcommand{\spd}[2]{\frac{\partial^2{#1}}{\partial{#2}^2}} %second partial derivative%
\newcommand{\mpd}[3]{\frac{\partial^2{#1}}{\partial{#2}\partial{#3}}}
%mixed partial derivatives


%%% Postavke dokumenta

%% URL u istom fontu kao i ostatak
\urlstyle{rm}

%% Dubina tablice sadržaja
\setcounter{tocdepth}{2}
%\setcounter{secnumdepth}{8}

% Definiramo razmak za cijeli dokument
\onehalfspacing

% Modifikacija Sadržaja - opcija titletoc
\usepackage{titletoc}
\titlecontents{part}
[3pc]
{\addvspace{1.5pc}\filcenter\hrule height 1pt \Large{\textbf{Partie}}~}
{\Large\textbf}
{\Large\textbf}
{}
[\hrule height 1pt\addvspace{.5pc}]




%% Briše footer i header na praznim stranicama u paru
\let\origdoublepage\cleardoublepage
\newcommand{\clearemptydoublepage}{%
    \clearpage
    {\pagestyle{empty}\origdoublepage}%
}
\let\cleardoublepage\clearemptydoublepage

%% Definicije i teoremi ako je potrebno

\newtheoremstyle{dotless}{}{}{\itshape}{}{\bfseries}{}{ }{}

\theoremstyle{dotless}
\newtheorem*{mydef*}{Definicija}

%% povišeni \chi da ne bude dolje

\DeclareRobustCommand{\rchi}{{\mathpalette\irchi\relax}}
\newcommand{\irchi}[2]{\raisebox{\depth}{$#1\chi$}} % inner command, used by \rchi



