\chapter{Virasoro algebra}\label{cha:Virasoro}

Virasoro algebra je centralna ekstenzija Wittove algebre. Ta ekstenzija je (slijedimo izvod dan u \citep{de1998lie}) jedinstvena do na izbor nekog parametra $c$, tako da je Virasoro algebra u biti klasa izomorfnih Liejevih algebra. Općenita centralna ekstenzija Wittove algebre se može zapisati kao\footnote{Podrazumijeva se da vrijedi $[L_m,c]=0$.}

\begin{equation}
[L_m,L_n]=(m-n)L_{m+n}+c_{m,n}
\label{eq:virasorogener}
\end{equation}

\noindent definirana sa setom konstanti $c_{m,n}$. Iz Jacobijevog identiteta imamo

\begin{equation}
[[L_m,L_n],L_p]+[[L_n,L_p],L_m]+[[L_p,L_m],L_n]=0
\label{eq:Jacobi}
\end{equation}

\noindent Svaki član možemo raspisati koristeći \eqref{eq:virasorogener}

\begin{equation}
\begin{split}
[[L_m,L_n],L_p]&=[(m-n)L_{m+n}+c_{m,n},L_p]=\\
&=(m-n)[L_{m+n},L_p]=\\
&=(m-n)\left((m+n-p)L_{m+n+p}+c_{m+n,p}\right)
\end{split}
\end{equation}

\noindent to napravimo za svaki član te vratimo u Jacobijev identitet \eqref{eq:Jacobi}

\begin{equation}
\begin{split}
& L_{m+n+p}\overbrace{\left((m+n)(m+n-p)+(n-p)(n+p-m)+(p-m)(p+m-n)\right)}^{=0}+\\
& +(m-n)c_{m+n,p}+(n-p)c_{n+p,m}+(p-m)c_{p+m,n}=0
\end{split}
\end{equation}

\noindent što nam daje

\begin{equation}
(m-n)c_{m+n,p}+(n-p)c_{n+p,m}+(p-m)c_{p+m,n}=0
\label{eq:koefJacob}
\end{equation}

\noindent Ako transformiramo generatore algebre tako da dodamo konstantu, sama algebra se neće promijeniti

\begin{equation}
L_m\rightarrow L_m'=L_m+b(m)
\end{equation}

\noindent To se može pokazati

\begin{equation*}
\begin{split}
[L_m',L_n'] &=[L_m+b(m),L_n+b(n)]=[L_m+b(m),L_n]+\overbrace{[L_m+b(m),b(n)]}^{=0}=\\
&=[L_m,L_n]=(m-n)L_{m+n}+c_{m,n}=\\
&=(m-n)(L_{m+n}'-b(m+n))+c_{m,n}=\\
&=(m-n)L_{m+n}'-b(m+n)(m-n)+c_{m,n}=\\
&=(m-n)L_{m+n}'+c_{m,n}'
\end{split}
\end{equation*}

\noindent pri čemu je 

\begin{equation*}
c'_{m,n}=c_{m,n}-b(m+n)(m-n)=c_{m,n}+c_{m,n}^{cob}
\end{equation*}

\noindent Dobili smo originalan oblik algebre uz redefiniciju konstanti $c_{m,n}$. Uzimajući to u obzir, uz definicije

\begin{equation*}
b(m)=\frac{1}{m}c_{m,0} \quad (m\neq 0),\quad 
b(0)=\frac{1}{2}c_{1,-1}
\end{equation*}

\noindent nam je

\begin{equation*}
\begin{split}
c_{0,n}'&=c_{0,n}+c_{0,n}^{cob}=c_{0,n}+\frac{1}{n}c_{n,0}\cdot n=c_{0,n}+c_{n,0}=0,\ (n\neq 0)\\
c_{1,-1}'&=c_{1,-1}+c_{1,-1}^{cob}=c_{1,-1}-\frac{1}{2}c_{1,-1}\cdot 2=0,
\end{split}
\end{equation*}

\noindent gdje smo iskoristili činjenicu da je Virasoro algebra kompleksna Liejeva algebra i moraju vrijediti antikomutacijske relacije $[L_m,L_n]=-[L_n,L_m]$. Stoga smo, bez gubitka općenitosti, mogli izabrati $c_{0,n}=0$ i $c_{1,-1}=0$ od samog početka. Ako stavimo $p=0$, jednadžba \eqref{eq:koefJacob} nam postaje

\begin{equation*}
(m-n)\underbrace{c_{m+n,0}}_{=-c_{0,m+n}=0}+n\underbrace{c_{n,m}}_{=-c_{m,n}}-mc_{m,n}=0\Rightarrow (m+n)c_{m,n}=0
\end{equation*}

\noindent Pošto je $m+n\neq 0$ mora biti $c_{m,n}=0$, no kad je $m+n=0$, odnosno $m=-n$ vrijedi 

\begin{equation}
c_{m,n}=c(m,n)=c(m)\delta_{m,-n}
\end{equation}

\noindent te zbog antisimetrije koeficijenta $c_{m,n}$ vrijedi da je $c(-m)=-c(m)$.

\noindent Funkcionalan oblik $c(m)$ se može naći ako se ponovo vratimo Jacobijevom identitetu, no sada sa oblikom algebre

\begin{equation}
[L_m,L_n]=(m-n)L_{m+n}+c(m)\delta_{m,-n}
\end{equation}

\noindent Tada imamo

\begin{equation*}
\begin{split}
&(m-n)[L_{m+n},L_p]+(n-p)[L_{n+p},L_m]+(p-m)[L_{p+m},L_n]=0\\[1ex]  
&(m-n)\left[(m+n-p)L_{m+n+p}+c(m+n)\delta_{m+n,-p}\right]+\\
&(n-p)\left[(n+p-m)L_{m+p+n}+c(n+p)\delta_{n+p,-m}\right]+\\
&(p-m)\left[(p+m-n)L_{p+m+n}+c(p+m)\delta_{p+m,-n}\right]=0\\[1ex] 
&(m-n)c(m+n)\delta_{m+n,-p}+(n-p)c(n+p)\delta_{n+p,-m}+\\
&+(p-m)c(p+m)\delta_{p+m,-n}=0\\[1ex] 
&(m-n)c(m+n)\delta_{m+n,-p}+(n+m+n)c(n-m-n)+\\
&+(-n-m-m)c(-n-m+m)=0\\[1ex] 
&(m-n)c(m+n)\delta_{m+n,-p}+(m+2n)c(-m)+(-n-2m)c(-n)=0\\[1ex] 
&\left[(m-n)c(m+n)-(m+2n)c(m)+(n+2m)c(n)\right]\delta_{n+p+m,0}=0
\end{split}
\end{equation*}

\noindent odnosno 

\begin{equation*}
(m-n)c(m+n)-(m+2n)c(m)+(n+2m)c(n)=0
\end{equation*}

\noindent Za $n=1$ imamo

\begin{equation*}
\begin{split}
(m-1)c(m+1)&-(m+2)c(m)+(1+2m)c(1)=0\\
(m-1)c(m+1)&-(m+2)c(m)=0\\
c(m+1)&=\frac{m+2}{m-1}c(m),\ m\ge 2
\end{split}
\end{equation*}

\noindent Rješavanjem rekurzije dobijemo

\begin{equation*}
c(m+1)=\frac{1}{3!}\frac{(m+2)!}{(m-1)!}c(2)
\end{equation*}

\noindent ako definiramo $c(2)=c/2$ imamo

\begin{equation*}
c(m+1)=\frac{1}{12}(m+2)(m+1)m c
\end{equation*}

\noindent Konačno imamo (uz $m+1\to m$)

\begin{equation}
\begin{split}
c(m)=\frac{c}{12}m(m^2-1)&,\ m\ge 2\\
c(-m)=-c(m),\quad c(0)=&c(1)=c(-1)=0
\end{split}
\end{equation}

\noindent Treba još pokazati da je ova ekstenzija netrivijalna, odnosno da ne postoji $b(m)$, takav da je

\begin{equation*}
-(m-n)b(m+n)=\frac{c}{12}m(m^2-1)\delta_{m+n,0}
\end{equation*}

\noindent što nije moguće jer bismo za $m+n=0$ imali uvjet

\begin{equation*}
-2mb(0)=\frac{c}{12}m(m^2-1)
\end{equation*}

\noindent što očito ne može biti ispunjeno za svaki $m$. Analognom procedurom sa $\bar{L}_m$ generatorima dobili bismo set centralnih članova sa istim svojstivma, definiranim sa nekom novom konstantom $\bar{c}$. Pokazali smo da Virasoro algebra, centralna ekstenzija Wittove algebre ima sljedeća svojstva


\begin{equation}
\begin{split}
[L_m,L_n]&=(m-n)L_{m+n}+\frac{c}{12}m(m^2-1)\delta_{m+n,0}\\
[\bar{L}_m,\bar{L}_n]&=(m-n)\bar{L}_{m+n}+\frac{\bar{c}}{12}m(m^2-1)\delta_{m+n,0}\\
[L_m,\bar{L}_n]&=0
\end{split}
\end{equation}

\noindent Generatore $L_m$ i $\bar{L}_m$ možemo smatrati kao koeficijente u Laurentovom razvoju holomorfnog i antiholomorfnog tenzora energije-impulsa u dvije dimenzije. Zato se Virasoro algebra često opisuje kao kvantna verzija Wittove algebre ili obratno, Wittova algebra kao klasični limes Virasoro algebre.
