\chapter{Izvod NHEK metrike}\label{cha:izvodNHEK}

Zanima nas regija blizu horizonta Kerrove crne rupe u ekstremnom limesu, odnosno metrika blizu horizonta ekstremne (ili ekstremalne) Kerrove crne rupe.

\noindent Prvo uzmemo linijski element za Kerrovu crnu rupu u Boyer-Lindquistovim koordinatama:

\begin{equation}
ds^2=-\frac{\Delta}{\rho^2}\left(d\hat{t}-a\sin^2\theta d\theta d\hat{\phi}\right)^2+\frac{\sin^2\theta}{\rho^2}\left(\left(\hat{r}^2+a^2\right)d\hat{\phi}-a d\hat{t}\right)^2+\frac{\rho^2}{\Delta}d\hat{r}^2+\rho^2d\theta^2
\end{equation}

\noindent Pri čemu su pokrate

\begin{equation}
\Delta\equiv\hat{r}^2-2M\hat{r}+a^2,\quad \rho^2\equiv\hat{r}^2+a^2\cos^2\theta,\quad a\equiv\frac{G J}{M},\quad M\equiv M_{ADM}
\end{equation}

\noindent Sljedeći originalan članak za dobivanje takozvane 'geometrije grla' Kerrove crne rupe \citep{Bardeen:1999px}, možemo metriku zapisati kao

\begin{equation}
ds^2=-e^{-2\nu}d\hat{t}^2+e^{2\psi}(d\hat{\phi}-\omega d\hat{t})^2+\rho^2\left(\Delta^{-1}d\hat{r}^2+d\theta^2\right)
\label{eq:BarHormetric}
\end{equation}

\noindent Gornje definicije za $\rho$ i $\Delta$ vrijede, no sada su nam 

\begin{equation}
e^{2\nu}\equiv\frac{\Delta\rho^2}{(\hat{r}^2+a^2)^2-\Delta a^2\sin^2\theta},\quad e^{2\psi}\equiv\Delta\sin^2\theta e^{-2\nu},\quad \omega\equiv\frac{2M\hat{r}a}{\Delta\rho^2}e^{2\nu}
\end{equation}

\noindent Ukupna masa je $M$, zakretni impuls je $J=Ma$, u sustavu smo gdje je $G=1$, te je $a=GJ/M$.

\noindent Za ekstremalni limit je 

\begin{equation}
a^2=M^2,\quad \Delta=(\hat{r}-M)^2
\end{equation}

\noindent Horizont događaja dobivamo tako da pustimo $g_{rr}\to\infty$ odnosno $g^{rr}=0$ te dobivamo $\hat{r}=M$. Vrijednost $\omega$ na horizontu se naziva kutna brzina horizonta te ju možemo jednostavno izvesti

\begin{equation}
\begin{split}
\omega&=\frac{2M\hat{r}a}{\Delta\rho^2}e^{2\nu}=\frac{2M\hat{r}a}{\Delta\rho^2}\cdot\frac{\Delta\rho^2}{(\hat{r}^2+a^2)^2-\Delta a^2\sin^2\theta}=\\
&=\frac{2M\hat{r}a}{(\hat{r}^2+a^2)^2-\Delta a^2\sin^2\theta}\xrightarrow{extrem.}\frac{2M^3}{4M^4}=\frac{1}{2M}
\end{split}
\end{equation}

\noindent Da bismo opisali geometriju blizu horizonta uvodimo koordinate

\begin{equation}
\hat{r}=M+\lambda r,\quad \hat{t}=\frac{t}{\lambda},\quad \hat{\phi}=\phi+\frac{t}{2M\lambda}
\end{equation}

\noindent i uzimamo limit $\lambda\to 0$. Pomak sa $\hat{\phi}$ na $\phi$ čini $\partial/\partial t$ tangentnim horizontu, odnosno, koordinate korotiraju sa horizontom.\medskip

\noindent Pišemo diferencijale te gledamo dobiveni linijski element

\begin{equation}
d\hat{r}=\lambda dr,\quad d\hat{t}=\frac{dt}{\lambda},\quad d\hat{\phi}=d\phi+\frac{dt}{2M\lambda}
\end{equation}

\noindent Za prvi dio metrike \eqref{eq:BarHormetric} imamo

\begin{equation}
\begin{split}
-e^{2\nu}d\hat{t}^2&=-\frac{\lambda^2 r^2\left(M^2(1+\cos^2\theta)+\lambda^2 r^2+2\lambda r M\right)}{(2M^2+\lambda^2 r^2+2M\lambda r)^2-\lambda^2 r^2 M^2\sin^2\theta}\cdot\frac{dt^2}{\lambda}\\
&\xrightarrow{\lambda\to 0}-\frac{M^2 r^2(1+\cos^2\theta)}{4M^4}dt^2=-\frac{1+\cos^2\theta}{2}\cdot\frac{r^2}{2M^2}dt^2=\\
&=\Big\{\textrm{uz zamjenu}\ 2M^2\equiv r_0^2\Big\}=-\left(\frac{1+\cos^2\theta}{2}\right)\frac{r^2}{r_0^2}dt^2
\end{split}
\end{equation}

\noindent Zatim gledamo drugi dio. Predfaktor $e^{2\psi}$ je

\begin{equation}
\begin{split}
e^{2\psi}&=\Delta\sin^2\theta e^{2\nu}=\Delta \sin^2\theta\cdot\frac{(\hat{r}^2+a^2)^2-\Delta a^2\sin^2\theta}{\Delta\rho^2}\\
&\xrightarrow{\lambda\to 0}\sin^2\theta\cdot\frac{4M^4}{M^2(1+\cos^2\theta)}=\frac{2r_0^2\sin^2\theta}{1+\cos^2\theta}
\end{split}
\end{equation}

\noindent Da bismo riješili zagradu, moramo vidjeti kakva je ovisnost $\omega$ o $\lambda$:

\begin{equation}
\begin{split}
\omega&=\frac{2M\hat{r}a}{(\hat{r}^2+a^2)^2-\Delta a^2\sin^2\theta}=\\
&=\frac{2M^2(M+r\lambda)}{\left((M+r\lambda)^2+M^2\right)^2-\left((M+r\lambda)^2-2M(M+r\lambda)+M^2\right)M^2\sin^2\theta}=\\
&=\frac{2M^3\left(1+\frac{r\lambda}{M}\right)}{\left(4M^4+8M^2r^2\lambda^2+r^4\lambda^4+8M^3r\lambda+4Mr^3\lambda^3\right)-M^2r^2\lambda^2\sin^2\theta}=\\
&=\frac{2M^3\left(1+\frac{r\lambda}{M}\right)}{4M^4\left(1+\frac{2r\lambda}{M}+\lambda\bigo(\lambda)\right)}=\\
&=\Big\{\textrm{Nazivnik razvijemo oko}\ \lambda\to 0\Big\}=\frac{1-\frac{r\lambda}{M}+\lambda\bigo(\lambda)}{2M}
\end{split}
\end{equation}

\noindent Pa nam zagrada postaje

\begin{equation}
\begin{split}
(d\hat{\phi}-\omega d\hat{t})^2&=\left(d\phi+\frac{dt}{2M\lambda}-\frac{1}{2M}\left(\left(1+\frac{r\lambda}{M}\right)+\lambda\bigo(\lambda)\right)\frac{dt}{\lambda} \right)^2=\\
&=\left(d\phi+\cancel{\frac{dt}{2M\lambda}}-\cancel{\frac{dt}{2M\lambda}}-\frac{r dt}{2M^2}-\cancelto{0}{\bigo(\lambda)}\hspace*{1em} \right)^2=\\
&=\left(d\phi-\frac{r}{r_0^2}dt\right)^2
\end{split}
\end{equation}

\noindent Na kraju nam još ostaju

\begin{equation}
\begin{split}
\frac{\rho^2}{\Delta}d\hat{r}^2&=\frac{\hat{r}^2+M^2\cos^2\theta}{(\hat{r}-M)^2}\cdot \lambda^2dr^2=\\
&=\frac{(M+\lambda r)^2+M^2\cos^2\theta}{(M+\lambda r-M)^2}\cdot \lambda^2dr^2\xrightarrow{\lambda\to 0}\frac{M^2(1+\cos^2\theta)}{r^2}dr^2=\\
&=\left(\frac{1+\cos^2\theta}{2}\right)\frac{r_0^2}{r^2}dr^2
\end{split}
\end{equation}

\noindent i

\begin{equation}
\begin{split}
\rho^2d\theta^2&=(\hat{r}^2+M^2\cos^2\theta)d\theta^2=\left((M+\lambda r)^2+M^2\cos^2\theta\right)d\theta^2\xrightarrow{\lambda\to 0}\\
&=\left(\frac{1+\cos^2\theta}{2}\right)r_0^2d\theta^2
\end{split}
\end{equation}

\noindent Kada sve to skupa zapišemo dobijemo linijski element

\begin{equation}
ds^2=\left(\frac{1+\cos^2\theta}{2}\right)\left[-\frac{r^2}{r_0^2}dt^2+\frac{r_0^2}{r^2}dr^2+r_0^2d\theta^2\right]+\frac{2r_0^2\sin^2\theta}{1+\cos^2\theta}\left(d\phi+\frac{r}{r_0^2}dt\right)^2
\label{eq:metrika2}
\end{equation}

\noindent Ovakvo definirano prostor-vrijeme nije asimptotski ravno. Ako stavimo $\theta=0$ ili $\theta=\pi$, metrika postane $AdS_2$. Također, osim $\partial/\partial t$ i $\partial/\partial \phi$ simetrija, \eqref{eq:metrika2} je invarijantna na $r\to Cr$, $t\to t/C$ za bilo koju konstantu $C$. To nam daje dilatacijsku simetriju $AdS_2$. Da bismo pokazali da je \eqref{eq:metrika2} invarijantna na (analogonom) globalne vremenske translacije $AdS_2$.

\noindent Zato uvodimo nove koordinate koje su povezane sa ($r,t$) na isti način kao što su globalne koordinate $AdS_2$ povezane sa Poincar\'eovim koordinataama. Uz $r_0=1$ imamo

\begin{equation}
r=(1+y^2)^{1/2}\cos\tau+y,\quad t=\frac{(1+y^2)^{1/2}\sin\tau}{r}
\end{equation}

\noindent Dok novu aksijalnu koordinatu $\varphi$ izabiremo tako da je $g_{\varphi y}=0$

\begin{equation}
\phi=\varphi+\log\left|\frac{\cos\tau+y\sin\tau}{1+(1+y^2)^{1/2}\sin\tau}\right|
\end{equation}

\noindent Diferenciranjem dobijemo diferencijale

\begin{equation}
dr=\left(1+\frac{y\cos\tau}{(1+y^2)^{1/2}}\right)dy-(1+y^2)^{1/2}\sin\tau d\tau
\end{equation}

\begin{equation}
dt=\frac{1+y^2+y(1+y^2)^{1/2}\cos\tau}{(y+(1+y^2)^{1/2}\cos\tau)^2}d\tau-\frac{\sin\tau}{(1+y^2)^{1/2}(y+(1+y^2)^{1/2}\cos\tau)^2}dy
\end{equation}

\begin{equation}
\begin{split}
d\phi&=d\varphi-\frac{(1+y^2)^{1/2}-y\cos\tau+\sin\tau}{(\cos\tau+y\sin\tau)\left(1+(1+y^2)^{1/2}\sin\tau\right)}d\tau+\\
&+\frac{\sin\tau\left((1+y^2)^{1/2}-y\cos\tau+\sin\tau\right)}{(\cos\tau+y\sin\tau)\left((1+y^2)^{1/2}+(1+y^2)\sin\tau\right)}dy
\end{split}
\end{equation}

\noindent Uvrštavanjem u \eqref{eq:metrika2} dobili smo novi linijski element

\begin{equation}
ds^2=\left(\frac{1+\cos^2\theta}{2}\right)\left[-(1+y^2)d\tau^2+\frac{dy^2}{1+y^2}+d\theta^2\right]+\frac{2\sin^2\theta}{1+\cos^2\theta}(d\varphi+yd\tau)^2
\label{eq:NHEKmetrika}
\end{equation}

\noindent Ovakvo rješenje, poznato pod nazivom 'geometrija grla' (engl. \textit{throat geometry}), ima sve simetrije $AdS_2$ uz translacije u $\varphi$: grupa izometrija metrike \eqref{eq:NHEKmetrika} je $SL(2,R)\times U(1)$. Sve geometrijske veličine ovise samo o $\theta$. 

\noindent Koordinate u \eqref{eq:NHEKmetrika} (s $\tau\in \langle -\infty,\infty\rangle$, $y\in \langle -\infty,\infty\rangle$) pokrivaju čitavo prostor-vrijeme. Površine konstantnog $\tau$ su uvijek prostornog tipa pa $\tau$ možemo identificirati kao globalnu vremensku funkciju, što znači da ovakvo prostor-vrijeme nema zatvorene vremenske krivulje.

\noindent No, Killingov vektor $\partial/\partial\tau$ nije vremenskog tipa posvuda. Vremenskog tipa je za svaki $\theta$ kada je $y^2<1/3$, ali je asimptotski prostornog tipa za $\sin\theta>(1+\cos^2\theta)/2$ odnosno $\sin\theta>0.536$, unutar $32.4^\circ$ ekvatorijalne ravnine. 

\noindent To je posljedica rotacije i analogna je ergosferi u slučaju ekstremne Kerrove crne rupe.