\section{Algebra difeomorfizama}\label{sec:Diff}

Detaljno ćemo izvesti račun iz članka \citep{Guica:2008mu}. Krećemo od metrike blizu horizonta ekstremne Kerrove crne rupe \eqref{eq:NHEKlinel}

\begin{equation*}
d\bar{s}^2=2GJ\Omega^2\left(-(1+r^2)d\tau^2+\frac{dr^2}{1+r^2}+d\theta^2+\Lambda^2(d\varphi+rd\tau)^2\right)
\end{equation*}

\noindent Pošto nemamo asimptotski ravno prostor-vrijeme na granici $r\to\infty$, nisu nam poznati rubni uvjeti, kao što je to slučaj s AdS${}_3$ \citep{Brown:1986nw}. Stoga ih moramo sami zadati te provjeriti da li će nam dobiveni difeomorfizmi dati naboje koji su konačni i integrabilni.

Cilj je, preko zadanih rubnih uvjeta konstruirati asimptotski Killingov vektor $\xi$ koji, uz izbor funkcija ovisno o simetrijama, zadovoljava Virasoro algebru.

\noindent Radimo u bazi ($\tau,\ r,\ \theta,\ \phi$) te pišemo rubne uvjete

\begin{equation}
\bordermatrix{~ & \tau & r & \theta & \varphi \cr
              & h_{\tau\tau}=\bigo(r^2) & h_{\tau r}=\bigo(r^{-2}) & h_{\tau\theta}=\bigo(r^{-1}) & h_{\tau\varphi}=\bigo(1) &\cr
              & h_{r\tau}=\bigo(r^{-2}) & h_{rr}=\bigo(r^{-3}) & h_{r\theta}=\bigo(r^{-2}) & h_{r\varphi}=\bigo(r^{-1}) & \cr
              & h_{\theta\tau}=\bigo(r^{-1}) & h_{\theta r}=\bigo(r^{-2}) & h_{\theta\theta}=\bigo(r^{-1}) & h_{\theta\varphi}=\bigo(r^{-1}) \cr
              & h_{\varphi\tau}=\bigo(1) & h_{\varphi r}=\bigo(r^{-1}) & h_{\varphi\theta}=\bigo(r^{-1}) & h_{\varphi\varphi}=\bigo(1) \cr}
\label{eq:Guicaru}
\end{equation}

\noindent $h_{\mu\nu}=h_{\nu\mu}$ je devijacija metrike od pozadinske NHEK metrike $\bar{g}$ \eqref{eq:NHEKmet}\medskip

\noindent Rješavamo asimptotsku Killingovu jednadžbu

\begin{equation}
\Lie_\xi\bar{g}_{\mu\nu}=h_{\mu\nu}
\end{equation}

\noindent Zbog činjenice da je ovo jednadžba za simetričan tenzor, imamo 10 nezavisnih linearnih parcijalnih diferencijalnih jednadžbi prvog reda za vektor $\xi=\xi^\mu\partial_\mu$ \footnote{To bi za simetričan tenzor u D dimenzija dalo D(D+1)/2 jednadžbe.}.

\noindent Pretpostavljamo da nam komponente željenog difeomorfizma imaju ansatz

\begin{equation}
\xi^\mu=\sum\limits_{n=-1}^{\infty}\xi^\mu_n(\tau,\theta,\varphi)r^{-n}
\label{eq:ansatz}
\end{equation}

\noindent Uvrštavajući pretpostavljeni ansatz, nakon poduljeg računa danog u dodatku \ref{cha:IzvodDiff} dobijemo najopćenitiji difeomorfizam koji čuva rubne uvjete \eqref{eq:Guicaru}

\begin{equation}
\xi=[C+\mathcal{O}(r^{-3})]\partial_\tau+[-r\epsilon'(\varphi)+\mathcal{O}(1)]\partial_r+\mathcal{O}(r^{-1})\partial_\theta+[\epsilon(\varphi)+\mathcal{O}(r^{-2})]\partial_\varphi
\label{eq:diffeomorfizam}
\end{equation}

\noindent gdje je $\epsilon(\varphi)$ proizvoljna glatka funkcija, a $C$ je proizvoljna konstanta. Članovi nižeg reda će odgovarati trivijalnim difeomorfizmima. Grupa asimptotskih simetrija sadrži jednu kopiju konformne cikličke grupe generirane vektorom

\begin{equation}
\xi_\epsilon=\epsilon(\varphi)\partial_\varphi-r\epsilon'(\varphi)\partial_r
\label{eq:difeojedn}
\end{equation}

Jedan od fizikalnih razloga za ovakav izgled difeomorfizma jest taj što imamo ne nula temperaturu koja je asocirana s modovima koji korotiraju sa crnom rupom. Odnosno duž horizonta crne rupe opažači se gibaju brzinom svjetlosti. Postojanje temperature ukazuje na postojanje ekscitacija duž $\partial_\varphi$. Pošto imamo $\varphi\sim\varphi+2\pi$, možemo definirati $\epsilon_n(\varphi)=-e^{-in\varphi}$ i $\xi_n=\zeta(\epsilon_n)$.

\noindent Ovakvi generatori simetrija će zadovoljavati Virasoro algebru \ref{cha:Virasoro}

\begin{equation}
i[\xi_m,\xi_n]=(m-n)\xi_{m+n}
\end{equation}

\noindent Ako stavimo $n=0$ imamo 

\begin{equation}
\xi_0=-\partial_\varphi
\end{equation}

\noindent što je upravo generator $U(1)$ rotacijske izometrije.\medskip

Treba uočiti da ovakav odabir rubnih uvjeta ne daje generatore koji će pojačati\newline $SL(2,\mathbb{R})$ u Virasoro algebru. 

\noindent Generator $\tau$ translacija je $\partial_\tau$. Konjugirana očuvana veličina, očuvani naboj, tog generatora je veličina E${}_R$ koja mjeri devijaciju  od ekstrema ($M^2/G-J$) crne rupe. Pošto razmatramo ekstremne crne rupe ($J=M^2$) tada je E${}_R=0$, odnosno uzet je dodatan zahtjev $Q_{\partial_\tau}=0$. U članku od Guice et. al.  \citep{Guica:2008mu} je dano pojašnjenje takvog odabira.

%\noindent To je lako za provjeriti. Prvo možemo vidjeti da difeomorfizam \eqref{eq:difeojedn} i $\partial_\tau$ komutiraju $[\xi_\epsilon,\partial_\tau]=0$, zatim se može eksplicitno pokazati da je naboj koji mjeri devijaciju od ekstrema

%\begin{equation*}
%Q_{\partial_\tau}=\frac{1}{8\pi G}\int_{\partial\Sigma}k_{\xi_\epsilon}[\Lie_{\partial_\tau}g,g]=0
%\end{equation*}

%\noindent jer je Liejeva derivacija pozadinske metrike duž $\partial_\tau$ jednaka nuli. Stoga vidimo da je uvjet

%\begin{equation*}
%Q_{\partial_\tau}[g]=0
%\end{equation*}

%\noindent ispunjen.


\noindent U člancima \citep{Matsuo:2009sj, Matsuo:2009pg} uz drugačiji izbor rubnih uvjeta dobiju difeomorfizam

\begin{equation}
\xi=[\epsilon_\tau(\tau)+\mathcal{O}(r^{-1})]\partial_\tau+[-r\epsilon'_\tau(\tau)-r\epsilon_\varphi'(\varphi)+\mathcal{O}(1)]\partial_r+\mathcal{O}(r^{-1})\partial_\theta+[\epsilon_\varphi(\varphi)+\mathcal{O}(r^{-1})]\partial_\varphi
\end{equation}

\noindent Uz izbor $\epsilon_n(\varphi)=e^{-i n\varphi}$ i $\epsilon_n(\tau)=\tau^{1+n}$ dobiju se generatori

\begin{equation}
l_n=e^{-in\varphi}\partial_\varphi+i nre^{-in\varphi}\partial_r\qquad \bar{l}_n=-i\tau^{1+n}+i(1+n)r\tau^n\partial_r
\end{equation}

\noindent koji zadovoljavaju Virasoro algebre

\begin{equation}
i[l_n,l_m]=(n-m)l_{n+m},\qquad i[\bar{l}_n,\bar{l}_m]=(n-m)\bar{l}_{n+m}
\end{equation}

Dvije algebre možemo shvatiti kao algebre lijevih i desnih modova (engl. \textit{movers}) u CFT. Također za $n=0$ ponovo dobijemo generator $U(1)$ izometrije $l_0$, dok za $n=-1,0,1$ za $\bar{l}_n$ dobijemo $SL(2,\mathbb{R})$ Killingove vektore.

Treba naglasiti da ako uključimo difeomorfizme koji dopuštaju pojačavanje $SL(2,\mathbb{R})$ grupe izometrija u Virasoro, dobijemo trivijalne naboje (koje ćemo izračunati za \eqref{eq:diffeomorfizam} i pokazati da nisu 0).

\noindent U slučaju ekstremnih geometrija nas ne zanima netrivijalna $SL(2,\mathbb{R})$ ili druga Virasoro albebra, jer je entropija ekstremnih crnih rupa dana koristeći samo jednu kopiju Virasoro algebre, dok druga kopija mjeri odstupanje od ekstrema crne rupe.

\noindent Jedna od mogućnosti je da se konstruira regija blizu horizonta za ne ekstremne crne rupe preko perturbacijske teorije, kao velika deformacija geometrije blizu horizonta ekstremne crne rupe.

\noindent U tom slučaju se algebra difeomorfizama, koja proširuje $SL(2,\mathbb{R})$ algebru na Virasoro, dobiva preko renormalizabilnog tenzora energije-impulsa (koji je ekvivalentan kvazi-lokalnom Brown-York tenzoru energije-impulsa). No takav opis više ne spada u standardnu asimptotsku analizu.




