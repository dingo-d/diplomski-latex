\chapter{Zaključak}\label{cha:Zaklj}

\lettrine[lines=4]{S} {umirajmo} najbitnije rezultate ovog diplomskog rada. Promatrali smo jedan vid AdS/CFT korespondencije, Kerr/CFT korespondenciju. Posebnost ovog vida korespondencije leži u tome što se ne oslanja na teoriju struna, supersimetrije ili neke druge vidove realizacije kvantne teorije gravitacije, nego se isključivo bazira na simetrijskim svojstvima promatrane geometrije. Prednost takvog pristupa je zasigurno jednostavniji račun kojeg smo morali provesti, no postoje i neke mane. Jedna od najvećih mana jest činjenica da je odabir rubnih uvjeta u prostornoj beskonačnosti proizvoljan. Glavni problem je tad naći odgovarajuće rubne uvjete koji će dati netrivijalne difeomorfizme koji će biti generatori asimptotskih simetrija. Moguće rješenje bilo bi potpuno automatiziranje procesa provjere putem računala, no to nije trivijalan zadatak. Guica, Hartman, Song i Strominger \citep{Guica:2008mu} su našli jedne rubne uvjete koji pojačavaju $U(1)$ simetriju u Virasoro, što nam daje netrivijalan centralni naboj, nužan za račun mikroskopske entropije preko Cardyjeve formule. Ne možemo reći da su to jedini konzistentni rubni uvjeti koji daju korespondenciju, jer postoje i drugi rubni uvjeti \citep{Matsuo:2009sj, Matsuo:2009pg} koji će dati korespondenciju. To je očita mana, jer ostavlja velike proizvoljnosti prilikom definiranja teorije. No unatoč tome, činjenica da korespondencija postoji ima veliki značaj koji samo dodatno potvrđuje AdS/CFT korespondenciju i holografski princip.

U ovom radu smo detaljno razradili ovu temu. Krenuli smo od geometrije Kerrove crne rupe te smo pokazali kako doći do geometrije blizu horizonta. Nakon toga smo uzeli ekstremni limes. 

Iako je to idealizacija koja je uzeta da bi računi ispali jednostavniji, postoje indikacije da takve crne rupe i postoje. Jedan od primjera je supermasivna crna rupa GRS 1915+105 \citep{supermasiv1} sa masom $M=14.0\pm 4.4M_{\small{\astrosun}}$, koja ima bezdimenzionalni parametar spina $a_*=\frac{cJ}{GM^2}=(0.988\pm 0.003)$. Bezdimenzionalni parametar spina opisuje brzinu vrtnje crne rupe i raspon mu je od $a_*=0$ za Schwarzschildovu crnu rupu do $a_*=1$ za Kerrovu crnu rupu (maksimalna ekstremnost). Drugi primjer je crna rupa u središtu galaksije NGC 1365 čiji je bezdimenzionalan parametar spina $a_*>0.84$ sa 90\% razinom pouzdanosti. Eksperimentalni dokazi uvijek daju dodatnu težinu teorijskim istraživanjima.

Nakon toga smo proveli račun traženja asimptotskih Killingovih vektora za dane rubne uvjete te smo dobili traženi difeomorfizam. Zatim smo, koristeći kovarijantni formalizam Barnicha i Brandta \citep{Barnich:2001jy}, našli izraz za centralni naboj za danu geometriju. Centralni naboj iznosi $c_L=12J/\hbar$ te zajedno sa temperaturom $T_L=/2\pi$, dobivenu iz razmatranja Frolov-Thorne vakuuma, dobivamo izraz za entropiju preko Cardijeve formule. Ključ korespondencije je činjenica da su entropije dobivene preko termodinamike (Bekenstein-Hawking) i kvantne fizike (Cardy) jednake. To daje potvrdu cjelokupnog računa.

Iako je ovo poučan i opširan problem, on nije riješen. Postoje otvorena pitanja poput skrivenih simetrija, da li se korespondencija može proširiti na ne-ekstremne slučajeve i druga zanimljiva pitanja. U svakom slučaju radi se o problemu na kojem će se moći otkrivati novi uvidi u teoriju kvantne gravitacije. 



