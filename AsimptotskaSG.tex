\chapter{Grupa asimptotskih simetrija}\label{cha:ASG}

\lettrine[lines=4]{D} {ifeomorfna} invarijantnost je osnovna simetrija opće teorije relativnosti. U prostor-vremenima koji posjeduju asimptotsku granicu, javlja se podgrupa difeomorfizama koji tvore grupu asimptotskih simetrija (engl. \textit{asymptotic symmetry group} (ASG)), koja djeluje netrivijalno na asimptotsku granicu.

\noindent Po definicij, asimptotske simetrije su one baždarne transformacije koje ostavljaju promatrane konfiguracije polja asimptotski invarijantnim \citep{Brown:1986nw}. Detalji grupe asimptotskih simetrija će ovisiti o promatranom prostor-vremenu, dinamici te o nametnutim rubnim uvjetima. Formalizam koji koristimo je onaj koji su razvili Barnich i Brandt \citep{Barnich:2001jy}.

Rubni uvjeti su ograničenja na dopuštene konačne deformacije pozadinske metrike. Ako pozadinsku metriku označimo sa $\bar{g}_{\mu\nu}$ i deformaciju sa $h_{\mu\nu}$,  deformirana metrika je $g_{\mu\nu}=\bar{g}_{\mu\nu}+h_{\mu\nu}$.

Jedan od problema u Kerr/CFT korespondenciji je taj što nije a priori očito koje rubne uvjete moramo postaviti da bismo dobili odgovarajuće difeomorfizme koji će nam koristiti u računu centralnog naboja potrebnog za dokaz korespondencije (nemamo asimptotski ravno prostor-vrijeme). Oni uvelike ovise o fizikalnim kontekstu. 

\noindent Difeomorfizam generiran vektorom $\xi$ će transformirati pozadinsku metriku

\begin{equation*}
\bar{g}_{\mu\nu}+\Lie_\xi \bar{g}_{\mu\nu}=g_{\mu\nu}=\bar{g}_{\mu\nu}+h_{\mu\nu}
\end{equation*}

\noindent za neku deformaciju $h_{\mu\nu}$.

Da bismo definirali asimptotske simetrije pozadinskog prostor-vremena, moramo specificirati dozvoljene fluktuacije $h_{\mu\nu}$ i dozvoljene difeomorfizme $\xi$.

Način na koji definiramo ASG je sljedeći:

\begin{enumerate}
\item Definiramo \textit{granicu} prostor-vremena. 

Moramo definirati što znači `ići u beskonačnost'. Obično se uzima da nam radijalna koordinata ide u beskonačnost $r\to\infty$, dok su nam ostale koordinate fiksne.

\item Definiramo \textit{rubne uvjete} za pozadinsku metriku.

Jedan od načina je da pretpostavimo funkcionalnu formu rubnih uvjeta $h_{\mu\nu}$. Uzimamo da se $h_{\mu\nu}$ može razviti polinomno bar do drugog reda u $r$ kao 

\begin{equation*}
h_{\mu\nu}=h_{\mu\nu}^1r^m+h_{\mu\nu}^2r^{m-1}+o(r^{m-1})
\end{equation*}

U našem slučaju, rubni uvjeti specificiraju $h_{\mu\nu}\to \bigo(r^m)$ u beskonačnosti, za neki $r^m$, pri čemu je $m(\mu,\nu)$ cijeli broj.

\item Trebamo naći najopćenitiji oblik difeomorfizama $\xi$ koji čuvaju rubne uvjete (to jest, zadovoljava $\Lie_\xi g_{\mu\nu}=\bigo(r^m)$ za $g_{\mu\nu}$ koji zadovoljava rubne uvjete) i formiraju dobro definiranu algebru danu Lievim zagradama. Takvi difeomorfizmi su \textit{kandidati za asimptotske simetrije}. Podskup tih difeomorfizama će biti unaprijeđen u ASG (i odgovarajuću algebru u algebru asimptotskih simetrija), ako difeomorfizmi zadovoljavaju uvjete koji će biti navedeni. Kao i sa perturbacijama $h_{\mu\nu}$, ovaj korak uključuje pretpostavljanje forme $\xi$. Pretpostavljamo da se $\xi$ mogu razviti u red kao:

\begin{equation*}
\xi^\mu=r^{m_\mu}\tilde{\xi}^\mu +o(r^{m_\mu})
\end{equation*}

\noindent gdje je $\tilde{\xi}^\mu$ neka funkcija od koordinata koje nisu $r$\footnote{Često se uzima da je razvoj $\xi^\mu$ razvoj u padajućim potencijama koordinate $r$, odnosno $\xi^\mu=\sum_n\tilde{\xi}^\mu_n(\tau,\theta,\phi)r^{-n}$}.

\item Zatim računamo \textit{naboje} pridružene kandidatima asimptotskih simetrija.

Formalizam Barnicha i Brandta \citep{Barnich:2001jy} omogućuje nam definiranje naboja kao površinskih integrala duž granice $\partial\Sigma$ nul ili prostornog izreza (granica ne mora biti u beskonačnosti). Zato trebamo, umjesto Noetherine struje koja je ($n-1$) forma, ($n-2$) forme, koje se računaju lokalno kao funkcional polja u prostornoj beskonačnosti. Te ($n-2$) forme se nazivaju superpotencijali \citep{Silva:1998ii}.
Odgovarajući naboj za asimptotsku simetriju $\xi$ je 

\begin{equation}
Q_\xi[g,\bar{g}]=\int_{\partial\Sigma}k_\xi[h,\bar{g}]
\label{eq:naboj}
\end{equation}

\noindent gdje je $k_\xi$ ($n-2$) forma konstruirana iz lineariziranih jednadžbi gibanja za $h_{\mu\nu}$, a $n$ je broj dimenzija prostor-vremena.
Explicitan izraz za $k_\xi$ je

\begin{equation}
\begin{split}
k_\xi[h,\bar{g}]=&k^{[\nu\mu]}_\xi[h,\bar{g}](d^{n-2}x)_{\nu\mu}\\[2.5mm]
(d^{n-p}x)_{\mu_1\ldots\mu_p}:=&\frac{1}{p!(n-p)!}\epsilon_{\mu_1\ldots\mu_n}dx^{\mu_p+1}\wedge\ldots\wedge dx^{\mu_n}\\[2.5mm]
k_\xi^{[\nu\mu]}[h,\bar{g}]=&-\frac{\sqrt{-\bar{g}}}{16\pi}\left[\bar{D}^\nu(h\xi^\mu)+\bar{D}_\sigma(h^{\mu\sigma}\xi^\nu)+\bar{D}^\mu(h^{\nu\sigma}\xi_\sigma)+\right.\\[2mm]
+&\frac{3}{2}h\bar{D}^\mu\xi^\nu+\frac{3}{2}h^{\sigma\mu}\bar{D}^\nu\xi_\sigma+\frac{3}{2}h^{\nu\sigma}\bar{D}_\sigma\xi^\mu-(\mu\leftrightarrow\nu)\left.\right]
\end{split}
\label{eq:superpot}
\end{equation}



\noindent Indekse dižemo i spuštamo pozadinskom metrikom $\bar{g}_{\mu\nu}$, $h=\bar{g}^{\mu\nu}h_{\mu\nu}$ je trag, dok je $\bar{D}^\mu$ pozadinska kovarijantna derivacija.

Općenito, izraz \eqref{eq:naboj} vrijedi samo za infinitezimalne perturbacije $h_{\mu\nu}$. Za konačne $h_{\mu\nu}$ moramo integrirati duž puta $\gamma$ u faznom prostoru da bismo mogli izračunati naboje. Zbog konzistentnosti, naboji moraju biti \textit{integrabilni}, što znači da su neovisni o putu $\gamma$. Dovoljan uvjet integrabilnosti je 

\begin{equation*}
\int_{\partial\Sigma}k_\xi[\delta h_1,g+\delta h_2]-k_\xi[\delta h_2,g+\delta h_1]-k_\xi[\delta h_1-\delta h_2,g]=0
\end{equation*}

za bilo koju metriku $g_{\mu\nu}$ dopuštenu s rubnim uvjetima. Integrabilnost je isto tako zadovoljena kada je svojstvo \textit{asimptotske linearnosit} ispunjeno

\begin{equation*}
Q_\xi[h,\bar{g}]=Q_\xi[h,\bar{g}+\delta g]
\end{equation*}

odnosno, naboj nema nelinearnih korekcija.
\item Trebamo odrediti da li su nam naboji \textit{konačni, integrabilni} i \textit{očuvani na ljusci}, gdje `na ljusci' podrazumijevamo da perturbacije zadovoljavaju linearizirane Einsteinove jednadžbe. 

Difeomorfizmi $\xi$ koji odgovaraju ne iščezavajućim nabojima, koji zadovoljavaju ove uvjete, su element ASG. Difeomorfizmi koji odgovaraju trivijalnim nabojima su trivijalne asimptotske simetrije.

ASG se sastoji od svih dopuštenih asimptotskih simetrija koje zadovoljavaju rubne uvjete za $h_{\mu\nu}$ i odgovaraju konačnim, integrabilnim i očuvanim nabojima, modulo trivijalnih asimptotskih simetrija. Vektori $\xi$ koji generiraju difeomorfizme su \textit{asimptotski Killingovi vektori}.

\begin{equation*}
ASG(\mathcal{M}):=\bigg\{\xi^\mu\ \Big\vert\ \Lie_\xi\ \textrm{poštuje rubne uvjete i}\ Q_\xi\neq 0\bigg\}
\end{equation*}

\item Računamo algebru naboja danu Diracovim zagradama. Algebra je definirana kao

\begin{equation*}
\{Q_{\xi_1},Q_{\xi_2}\}:=\delta_{\xi_2}Q_{\xi_1}=Q_{\xi_1}[\Lie_{\xi_2}g,\bar{g}]
\end{equation*}

te nam daje reprezentaciju algebre asimptotskih simetrija $\xi$, do na moguće centralne ekstenzije $\mathcal{K}_{\xi_1,\xi_2}[\bar{g}]$

\begin{equation*}
\{Q_{\xi_1},Q_{\xi_2}\}\simeq Q_{[\xi_1,\xi_2]}+\mathcal{K}_{\xi_1,\xi_2}[\bar{g}]-N_{[\xi_1,\xi_2]}[\bar{g}]
\end{equation*}

Gdje je $N_{[\xi_1,\xi_2]}[\bar{g}]$ proizvoljna normalizacijska konstanta koja je obično nula. U gornjoj formuli ``$\simeq$" označava jednakost na ljusci.
Eksplicitni izraz za centralni naboj $\mathcal{K}_{\xi_1,\xi_2}[\bar{g}]$ je

\begin{equation*}
\mathcal{K}_{\xi_1,\xi_2}[\bar{g}]=Q_{\xi_1}[\Lie_{\xi_2}\bar{g},\bar{g}]
\end{equation*}

Centralni naboj je netrivijalan ako se ne može apsorbirati u $N_{[\xi_1,\xi_2]}[\bar{g}]$. 

Rubni uvjeti su dio specifikacije ASG. Drugačiji rubni uvjeti mogu dati različite grupe asimptotskih simetrija. Problem je ako izaberemo preslabe rubne uvjete, generatori nam neće biti dobro definirani i integrali za računanje naboja će divergirati, no ako stavimo prerestriktivne rubne uvjete, teorija postaje trivijalna jer će nam
svi generatori iščeznuti.
\end{enumerate}

Ono što ćemo mi napraviti je reproducirati i objasniti postupak računa iz članka Guica et. al. \citep{Guica:2008mu}. Također treba napomenuti da su nađeni i drugi konzistentni rubni uvjeti \citep{Matsuo:2009sj, Matsuo:2009pg} koje ćemo komentirati kasnije.





